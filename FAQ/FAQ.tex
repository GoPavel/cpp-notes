\documentclass[12pt]{article}

\usepackage[utf8]{inputenc} % common
\usepackage[russian]{babel} % for russian lang

\usepackage{hyperref} % for link
\usepackage{graphicx} % for add picture
%\usepackage{daytime} % for displaying version number and date
\usepackage{datetime} % for current data
\usepackage{indentfirst} % Красная строка
\usepackage[usenames]{color} % for \textcolor
\usepackage{minted} % for highlight

% seting size page
\voffset=-20mm
\textheight=220mm
\hoffset=-25mm
\textwidth=180mm

\begin{document}
    \begin{center}

    {\Large \bf F.A.Q. для конспектов} \\
    \vspace{0.5em}
    {\large Собрано {\today} в {\currenttime}}

    Эта статья призвана помочь писать конспекты и стандартизировать их написания. Я не буду рассказвать все о LaTeX я только опишу инструменты, которыми стоит советую пользоваться, и которых я надеюсь вам будет достаточно.


    P. S. сложно добиться того, чтобы только из pdf получилась понятная статья, поэтому это статья-пример, которой слудует пользоваться так: смотрим код, потом смотрим во что это компилируется.

    \end{center}
\underline{\hbox to 1\textwidth{{ } \hfil{ } \hfil{ } }}

\tableofcontents

\newpage

\section{Установка LaTeX}

\subsection{Linux}

Вам нужно установить:
\begin{enumerate}
    \item texlive-live - сам LaTeX
    \item minted - для вставки исходного кода
    \item pygment - для minted

\textcolor{red}{NB}) Для сборки tex-файла нужно указывать в командной строке флаг \textbf(--shell-escape)
\end{enumerate}

\subsection{Win}
    TODO
\section{Работа с Git}
	\subsection{Начало работы с основным репрезиторем}
	TODO
	\subsection{Обновление локальной копии основного репрезитория}
	TODO
	\subsection{Залив в основной репрезиторий}
	TODO
\section{LaTeX}
	TODO

\subsection{Структура}

\begin{enumerate}
    \item Когда вы пишите статью, лучше всего делить ее на секции и подсекции, которые потом будут отражаться в оглавлении. Чем лучше структурирован текст, тем проще ориентироваться в оглавлении.
\end{enumerate}

\subsection{Вставка кода}

Для вставки исходного кода можно искользовать разные средства. Я остановился на \textbf{minted}. Мне показалось, что это очень удобная и красивая библиотека. Перейдем к примерам:


Вставим код:
\begin{minted}{c++}

    #include <iostream>

    using namespace std;

    int main() {
        cout << "Hello, world!" << endl;

        return 0;
    }
\end{minted}

Хорошо теперь наведем порядок, написав преамбулу.

\begin{minted}
    [linenos, % нумерация строк
     frame=lines, framesep=2mm, % черта сверху и снизу
     tabsize = 4, % размер табуляции
      breaklines % перенос текста
      ]{c++}

    #include <iostream>

    using namespace std;

    int main() {
        cout << "Hello, world!" << endl; // comment
        return 0;
    }
\end{minted}

Также можно вставлять код прямо в с текст \mintinline{c++}{std::shared_ptr; // smart pointer }. И я рекомендую делать именно так.


\textcolor{red}{NB}) Есть различные стили кода, но стандартная вроде не плоха.
\section{Советы и пожелания}
 TODO


\end{document}
